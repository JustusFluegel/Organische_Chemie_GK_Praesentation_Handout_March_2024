\begin{frame}
\frametitle{Table of Contents}
\tableofcontents
\end{frame}


\section{Intermolekulare Kräfte}
\subsection{Van-Der-Waals-Kräfte}
\begin{frame}
\frametitle{Intermolekulare Kräfte - Van-der-Waals-Kräfte}
\begin{itemize}
	\item<+-> Schwache Wechselwirkungen zwischen verschiedenen Atomen oder Molekülen
	\item<+-> Entstehung durch kurzzeitige Dipolmomente aufgrund ungleichmäßiger Elektronenverteilung um den Atomkern
	\item<+-> Unterteilt in drei Unterarten
\end{itemize}
\begin{block}<+->{Stärke}
	Van-der-Waals-Kräfte sind generell sehr schwache Kräfte
\end{block}
\end{frame}

\subsubsection{London-Kräfte}
\begin{frame}
\frametitle{Intermolekulare Kräfte - Van-der-Waals-Kräfte - London-Dispersion}
\begin{itemize}
	\item<+-> Spontane Polariation von Teilchen ($e^-$ \say{schwirren} gerade auf einer Seite)
	\item<+-> Induzierte Dipole in benachbarteten Teilchen
	\item<+-> Zwischen nicht-dipolen
	\item<+-> $\implies$ Teilchen ziehen sich an / stoßen sich ab
\end{itemize}
\begin{block}<+->{Stärke}
	Sehr schwach
\end{block}
\end{frame}

\subsubsection{Deybe-Wechselwirkung}
\begin{frame}
\frametitle{Intermolekulare Kräfte - Van-der-Waals-Kräfte - Deybe-Wechselwirkung (zwischen induzierten Dipolen)}
\begin{itemize}
	\item<+-> Bereits existierende Dipole in der Lösung
	\item<+-> Induzierte Dipole in benachbarteten Teilchen
	\item<+-> Zwischen Dipol und nicht-dipol
	\item<+-> $\implies$ Teilchen ziehen sich an / stoßen sich ab
\end{itemize}
\begin{block}<+->{Stärke}
	Sehr schwach, aber generell stärker als London-Kräfte
\end{block}

\end{frame}

\subsubsection{Dipol-Dipol-Kräfte}
\begin{frame}
\frametitle{Intermolekulare Kräfte - Van-der-Waals-Kräfte - Keesom-Kraft (Dipol-Dipol-Kräfte)}
\begin{itemize}
	\item<+-> Bereits existierende Dipole in der Lösung
	\item<+-> Besagte Dipole ziehen sich an / stoßen sich ab.
	\item<+-> Zwischen zwei Dipolen
\end{itemize}
\begin{block}<+->{Stärke}
	Sehr schwach, aber generell die stärkste der drei Van-der-Waals-Kräfte
\end{block}
\end{frame}

\subsection{Wasserstoffbrückenbindungen}
\begin{frame}
\frametitle{Intermolekulare Kräfte - Wasserstoffbrückenbindungen}
\chemfig{R^1-\charge{45:6pt=$\delta^-$}{X}-\charge{45:6pt=$\delta^+$}{H}-[:0,,,,dotted]\charge{180=\|,45:6pt=$\delta^-$}{Y}-R^2}
\begin{itemize}
	\item<+-> Zwischen Wasserstoffatom und stark elektronegativem Atom (\chemfig{O}, \chemfig{N}, \chemfig{F}, ...)
	\item<+-> Insbesondere an freiem Valenzelektronenpaar
\end{itemize}
\begin{block}<+->{Stärke}
	Schwächer als Ionenbindung, Kovalente Bindung, etc. aber deutlich stärker als Van-der-Waals-Kräfte
\end{block}
\begin{examples}[Beispiel]<+->
\chemfig{HO-H-[:0,,,,dotted]\charge{180=\|,90=\|}{O}H_2}
\end{examples}
\end{frame}

\section{Reaktionsmechanismen}
\subsection{Grundlagen}
\subsubsection{induktive Effekte}
\begin{frame}
\frametitle{induktive Effekte}
\begin{itemize}
	\item<+-> Elektronenverschiebungen entlang konvalenter Bindungen
	\item<+-> Bindungen mit Elektronegativtätsdifferenzen (aber keine Ionenbindung) (z.B. \ch{C-F})
	\item<+-> Beeinflusst Abspaltbarkeit von Teilmolekülen in z.B. nukleophiler Substitution ($S_N1$), elektrophiler Addition, ... (schwächt Bindung)
\end{itemize}
\pause
Zum Beispiel durch folgende Gruppen
\vspace*{-20pt}
\begin{columns}<+->
\begin{column}{0.48\textwidth}
\begin{block}<+->{$+\text{I}$-Effekt (Schiebend)}
\begin{itemize}
	\item<+-> t-Butylgruppe (\ch{-C(CH_3)_3})
	\item<+-> i-Propylgruppe (\ch{-CH(CH_3)_2})
	\item<+-> Alkylrest (\ch{-R})
\end{itemize}
% \begin[Beispiel]{examples}
% \end{examples}
\end{block}
\end{column}
\begin{column}{0.48\textwidth}
\begin{block}<+->{$-\text{I}$-Effekt (Ziehend)}
\begin{itemize}
	\item<+-> Hydroxygruppe (\ch{-OH}) / Carbonylgruppenteil (\ch{-C=O})
	\item<+-> Iodatom (\ch{-I}) / Bromatom (\ch{-Br}) / Chloratom (\ch{-Cl}) / Fluoratom (\ch{-F})
	\item<+-> Nitrogruppe (\ch{-NO_2}) / Aminogruppe (\ch{-NH_2}) / Carboxygruppe (\ch{-NH_2}) / Cyanogruppe (\ch{-CN}) / Sulfonylgruppe (\ch{-SO_3H})
\end{itemize}
% \begin{examples}[Beispiel]
% \end{examples}
\end{block}
\end{column}
\end{columns}
\end{frame}
\begin{frame}
\frametitle{Reaktionsmechanismen - induktive Effekte}
\begin{columns}
\begin{column}{0.48\textwidth}
\begin{block}{$+\text{I}$-Effekt (Schiebend)}
\begin{itemize}
	\item t-Butylgruppe (\ch{-C(CH_3)_3})
	\item i-Propylgruppe (\ch{-CH(CH_3)_2})
	\item Alkylrest (\ch{-R})
\end{itemize}
\end{block}
\begin{examples}[Beispiel]<+->
\chemfig{H_3\charge{[overlay=false]90:5pt=$\delta^-$}{C}-\charge{[overlay=false]90:5pt=$\delta^+$}{Li}}
\end{examples}
\end{column}
\begin{column}{0.48\textwidth}
\begin{block}{$-\text{I}$-Effekt (Ziehend)}
\begin{itemize}
	\item Hydroxygruppe (\ch{-OH}) / Carbonylgruppenteil (\ch{-C=O})
	\item Iodatom (\ch{-I}) / Bromatom (\ch{-Br}) / Chloratom (\ch{-Cl}) / Fluoratom (\ch{-F})
	\item Nitrogruppe (\ch{-NO_2}) / Aminogruppe (\ch{-NH_2}) / Carboxygruppe (\ch{-NH_2}) / Cyanogruppe (\ch{-CN}) / Sulfonylgruppe (\ch{-SO_3H})
\end{itemize}
\end{block}
\begin{examples}[Beispiel]<+->
\chemfig{H_3\charge{[overlay=false]90:5pt=$\delta^+$}{C}-\charge{[overlay=false]90:5pt=$\delta^-$}{Br}}
\end{examples}
\end{column}
\end{columns}
\end{frame}
\subsubsection{Reaktionsenthalpie}
% TODO: DO THIS
\begin{frame}
	\frametitle{Reaktionsenthalpie}
	// TODO
\end{frame}
\subsection{radikalische Substitution}
\begin{frame}
\frametitle{Reaktionsmechanismen - radikalische Substitution}
\begin{itemize}
	\item<+-> Wasserstoffatome werden von Alkanen abgespalten
	\item<+-> werden ersetzt/substituiert durch Halogenatome (Fluor (\ch{F}), Chlor (\ch{Cl}), Brom (\ch{Br}), Iod (\ch{I}))
	\item<+-> Benötigt zum Kettenstart externe Energie, um Radikale zu erzeugen (Sonnenlicht, Hitze, etc.)
\end{itemize}
\end{frame}
\begin{frame}
\frametitle{Reaktionsmechanismen - radikalische Substitution}
\begin{columns}
\begin{column}{0.36\textwidth}
\begin{examples}[Beispiel]
	\only<1-3>{
	\schemestart
		\chemfig{\charge{180=\|,90=\|,-90=\|}{Br}-\charge{0=\|,90=\|,-90=\|}{Br}}
		\arrow{->[*{0}$E_{\text{Light}}$][]}[-90, 0.8]
		2 \chemfig{\charge{180=\|,90=\|,-90=\|,0=.}{Br}}
	\schemestop
	}
	\only<4>{
	\schemestart
		\chemfig{\charge{180=\|,90=\|,-90=\|}{Br}-\charge{0=\|,90=\|,-90=\|}{Br}}
		\arrow{->[*{0}$E_{\text{Light}}$][]}[-90, 0.8]
		2 \chemfig{\charge{180=\|,90=\|,-90=\|,0=.}{Br}}
		\arrow{->[*{0}+ \chemfig{C_6H_13-[0,0.6]C(-[2,0.6]H)(-[-2,0.6]H)-[0,0.6]H}][]}[-90, 1]
	\schemestop
	}
	\only<5>{
	\schemestart
		\chemfig{\charge{180=\|,90=\|,-90=\|}{Br}-\charge{0=\|,90=\|,-90=\|}{Br}}
		\arrow{->[*{0}$E_{\text{Light}}$][]}[-90, 0.8]
		2 \chemfig{\charge{180=\|,90=\|,-90=\|,0=.}{Br}}
		\arrow{->[*{0}+ 2 \chemfig{C_6H_13-[0,0.6]C(-[2,0.6]H)(-[-2,0.6]H)-[0,0.6]H}][]}[-90, 1]
		2 \chemfig{C_6H_13-[0,0.6]\charge{0=.}{C}(-[2,0.6]H)(-[-2,0.6]H)} \+ 
		2 \chemfig{H-[0,0.6]\charge{[overlay=false]90=\|,0=\|,-90=\|}{Br}}
	\schemestop
	}
	\only<6>{
	\schemestart
		\chemfig{\charge{180=\|,90=\|,-90=\|,0=.}{Br}}
		\arrow{->[*{0}+ \chemfig{C_6H_13-[0,0.6]C(-[2,0.6]H)(-[-2,0.6]H)-[0,0.6]H}][]}[-90, 0.8]
		\chemfig{C_6H_13-[0,0.6]\charge{0=.}{C}(-[2,0.6]H)(-[-2,0.6]H)} \+ 
		\chemfig{H-[0,0.6]\charge{[overlay=false]90=\|,0=\|,-90=\|}{Br}}
		\arrow{->[*{0}+ \chemfig{\charge{180=\|,90=\|,-90=\|}{Br}-[0,0.6]\charge{0=\|,90=\|,-90=\|}{Br}}][]}[-90, 0.6]
		\chemfig{C_6H_13-[0,0.6]C(-[2,0.6]H)(-[-2,0.6]H)-[0,0.6]\charge{0=\|,90=\|,-90=\|}{Br}} \+ 
		\chemfig{\charge{180=\|,90=\|,-90=\|,0=.}{Br}}
	\schemestop
	}
	\only<7>{
	\schemestart
		\chemfig{@{start}\charge{180=\|,90=\|,-90=\|,0=.}{Br}}
		\arrow{->[*{0}+ \chemfig{C_6H_13-[0,0.6]C(-[2,0.6]H)(-[-2,0.6]H)-[0,0.6]H}][]}[-90, 0.8]
		\chemfig{C_6H_13-[0,0.6]\charge{0=.}{C}(-[2,0.6]H)(-[-2,0.6]H)} \+ 
		\chemfig{H-[0,0.6]\charge{[overlay=false]90=\|,0=\|,-90=\|}{Br}}
		\arrow{->[*{0}+ \chemfig{\charge{180=\|,90=\|,-90=\|}{Br}-[0,0.6]\charge{0=\|,90=\|,-90=\|}{Br}}][]}[-90, 0.6]
		\chemfig{C_6H_13-[0,0.6]C(-[2,0.6]H)(-[-2,0.6]H)-[0,0.6]\charge{0=\|,90=\|,-90=\|}{Br}} \+ 
		\chemfig{@{end}\charge{180=\|,90=\|,-90=\|,0=.}{Br}}
		\chemmove[-stealth,blue]{
			\draw ([yshift=3pt,xshift=3pt]end.45) .. controls +(350:20pt) and +(10:150pt) .. (start.45);
		}
	\schemestop
	}
\end{examples}
\end{column}
\begin{column}{0.60\textwidth}
\only<-7>{
\begin{block}{Kettenstart / Initation}
\begin{itemize}
	\item<+-> Hydrolytische Aufbrechung vom Brom
	\item<+-> Zuführung von Energie (Licht, Wärme)
	\item<+-> Bindungsparter behalten Elektronen
\end{itemize}
\end{block}
}
\begin{block}<+->{Kettenfortschritt / Folgereaktion / Prolongation}
\begin{itemize}
	\item Reaktion mit Kohlenwasserstoff
	\item<+-> Bildung weiterer Radikale, es entsteht \chemfig{H-\charge{[overlay=false]90=\|,0=\|,-90=\|}{Br}} und ein Alkylradikal
	\item<+-> Reaktion mit unreagierten Halogenmolekül, es entsteht ein Halogenalkan
	\item<+-> Wiederholen dieses Schrittes bis kein Edukt mehr vorliegt
\end{itemize}
\end{block}
\only<8->{
\begin{block}{Kettenabbruch}
\begin{itemize}
	\item
\end{itemize}
\end{block}
}
\end{column}
\end{columns}
\end{frame}
\subsection{elektrophile Addition}
\subsection{Eliminierung}
\subsection{nukleophile Substitution}
\begin{frame}
\frametitle{Reaktionsmechanismen - nukleophile Substitution - $S_N1$}
\begin{examples}[Beispiel]
% \setchemfig{ scheme debug = true}
\schemestart
	\chemfig{
		X-[@{bond}-2](-[-3]R^1)(<:[-1]R^2)<[-1.7]R^3 
	}
	\chemmove[red,-stealth]{
		\draw (bond.0) .. controls +(10:20 pt) and +(-80:10pt) .. ([yshift=18pt, xshift=17pt]bond.180);
	}
	\arrow{->}
	\chemfig{X^\fminus} \+
	\chemleft[
	\chemfig{R_1-@{anchormid}(<:[1]R_2)(<[-1]R_3)-[2,0.33,,,draw=none]\fscrp}
	\chemright]
	\arrow(aa--){->[+ \chemfig{^\fminus @{anchor1}Nu}]}[20]
	\chemmove[red,-stealth]{
		\draw ([yshift=12pt, xshift=-3pt]anchor1.110).. controls +(120:10pt) and +(150:80pt) ..(anchormid.90);
	}
	\chemfig{
		Nu-[-2](-[-3]R^1)(<:[-1]R^2)<[-1.7]R^3 
	}
	\arrow(@aa--){->[+ \chemfig{_\fminus @{anchor2}Nu}]}[-20]
	\chemmove[red,-stealth]{
		\draw ([yshift=-12pt, xshift=-3pt]anchor2.250).. controls +(240:10pt) and +(210:80pt) ..(anchormid.270);
	}
	\chemfig{
		Nu-[2](-[3]R^1)(<:[1]R^2)<[1.7]R^3 
	}
\schemestop
\end{examples}
\end{frame}
\begin{frame}
\frametitle{Reaktionsmechanismen - nukleophile Substitution - $S_N1$}
\begin{block}{Schritte}
Die $S_N1$-Reaktion verläuft 2-Schrittig
\end{block}
\begin{alertblock}{Reaktionsgeschwindigkeit}
Bei einer $S_N1$-Reaktion beeinflusst \textbf{nur eine} Konzentration die Reaktionsgeschwindigkeit (weil sie in 2 Schritten verläuft)
\begin{align*}
	v &= k_1 \cdot c\left[\text{Substrat}\right]
\end{align*}
\end{alertblock}
\end{frame}
\begin{frame}
\frametitle{Reaktionsmechanismen - nukleophile Substitution - $S_N2$}
\begin{examples}[Beispiel]
% \setchemfig{ scheme debug = true}
\schemestart
	\chemfig{
		X-@{midpoint_1}(-[1.2]R^1)(<:[-1.2]R^2)<[-0.5]R^3 
	}
	\arrow{->[+ \chemfig{^{\fminus}@{charge}Nu}]}
	\chemmove[red,-stealth]{
		\draw ([yshift=4pt, xshift=-13pt]charge.120) .. controls +(120:5 pt) and +(15:20pt) .. (midpoint_1.0);
	}
	\chemleft[
		\chemfig{@{x}X-[@{bond}:0,,,,dashed](-[:0,,,,dashed]Nu)(-[2]R^1)(<[-1]R^3)<:[-3]R^2}
	\chemright]
	\chemmove[-stealth,red]{
		\draw (bond.90).. controls +(100:20 pt) and +(80:10pt) .. (x.45) ;
	}
	\chemmove{\node[] at (-3pt,43pt) {\fscrm};}
	\arrow(m--x){->}
	\chemfig{X^\fminus} \+
	\arrow(@x--r){0}[,0]
	\chemfig{R^1-[-1.8](<[-2.2]R^3)(<:[-3.2]R^2)-Nu}
\schemestop
\end{examples}
\end{frame}
\begin{frame}
\begin{block}{Schritte}
Die $S_N2$-Reaktion verläuft 1-Schrittig
\end{block}
\begin{alertblock}{Reaktionsgeschwindigkeit}
Bei einer $S_N2$-Reaktion beeinflussen \textbf{beide} Konzentrationen die Reaktionsgeschwindigkeit (weil sie in einem Schritt verläuft)
\begin{align*}
	v &= k_2 \cdot c\left[\text{Substrat}\right] \cdot c\left[\text{Nukleophil}\right]
\end{align*}
\end{alertblock}
\end{frame}

\subsection{elektrophile Substitution}
\section{Stoffklassen}

% TODO: do this 
\begin{frame}
\frametitle{Stoffklassen}
// TODO
\end{frame}
